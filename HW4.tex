\documentclass{article}
\title{HW4}
\author{Brody Kendall}
\date{10/28/2021}
\usepackage{multirow}
\usepackage{centernot}
\begin{document}
\maketitle

\textbf{1. 3.3}\\

Using SAS, we find that the sample odds ratio is 0.7690 with a 95\% CI of (0.2862, 2.0659). Since this CI contains 1, it is plausible that the successive free throws are independent. In addition, we find using SAS that the $L^2 = 0.2858$ and $S^2 = 0.2727$, both yielding p-values of $0.6$. Again, this supports the claim that it is plausible that the successive free throws are independent.
\\

\textbf{2. 3.9(a)}\\

Standardized Pearson residuals:

\begin{tabular}{ |c|c|c|  }
 \hline
 \textbf{Diagnosis} & \textbf{Drugs} & \textbf{No Drugs}\\
 \hline
 \textbf{Schizophrenia} & 7.8745256774 & -7.8745256774\\
 \hline
 \textbf{Affective disorder} & 1.6022623041 & -1.6022623041\\
 \hline
 \textbf{Neurosis} & -2.385315371 & 2.385315371\\
 \hline
 \textbf{Personality disorder} & -4.84170066 & 4.84170066\\
 \hline
 \textbf{Special symptoms} & -5.139491197 & 5.139491197\\
 \hline
\end{tabular}\\\\

Higher magnitudes of standardized pearson residuals for schizophrenia, personality disorder, and special symptoms suggest a potential departure from independence for these cells. There are relatively smaller magnitudes for affective disorder and neurosis, but they are still not close to zero so we should not be confident of independence based on these values alone.
\\

\textbf{3. 3.12}\\

Using SAS, we find a 95\% CI for gamma to be $(0.3156, 0.4591)$. Since this interval does not contain 0, we can conclude that schooling and attitude toward abortion are not independent.
\\

\textbf{4. 3.15}\\

a. Using SAS, we find that a 95\% CI for the odds ratio using the Woolf interval is $(0, \infty)$ as one of the cells in the table has value 0.
\\

b. Again using SAS, we find that a 95\% CI for the odds ratio using Cornfield's "exact" approach is given by $(2.6460, \infty)$
\\

c. Again using SAS, we find that a 95\% CI for the odds ratio using the profile likelihood is given by $(5.117, \infty)$
\\

\textbf{5. 3.31}\\

a. $\pi_{1+} = \pi_{11} + \pi_{12} = \theta^2 + \theta(1-\theta) = \theta^2 + \theta - \theta^2 = \theta$

$\pi_{+1} = \pi_{11} + \pi_{21} = \theta$.

So, $\pi_{1+} = \pi_{+1}$ meaning the marginal distributions are identical and that independence holds ($\pi_{ij} = \pi_{i+}\pi_{+j}$, $i,j=1,2$)
\\

b. $L = 2Y_{11} log(\theta) + (Y_{12} + Y_{21}) log(\theta) + (Y_{12} + Y_{21}) log(1 - \theta) + 2Y_{22} log(1 - \theta)$, 

$\frac{\partial{L}}{\partial{\theta}} = 2Y_{11}/\theta + (Y_{12} + Y_{21})/\theta - (Y_{12} + Y_{21})/(1 - \theta) - 2Y_{22}/(1 - \theta)$. setting equal to 0, we get

$2Y_{11}(1-\theta) + (Y_{12} + Y_{21})(1-\theta) - (Y_{12} + Y_{21})\theta - 2Y_{22}\theta$

$= 2Y_{11}-2Y_{11}\theta + Y_{12} + Y_{21}-Y_{12}\theta - Y_{21}\theta - Y_{12}\theta - Y_{21}\theta - 2Y_{22}\theta  =0$

So $2Y_{11} + Y_{12} + Y_{21} = 2Y_{11}\theta + Y_{12}\theta + Y_{21}\theta + Y_{12}\theta + Y_{21}\theta + 2Y_{22}\theta$

And now we have $\hat\theta = (2Y_{11} + Y_{12} + Y_{21})/2(Y_{11} + Y_{12} + Y_{21} + Y_{22})$

$= (Y_{1+} + Y_{+1})/2n = (p_{1+} + p_{+1})/2$
\\

c. We can calculate $S^2 = \Sigma_i \Sigma_j \frac{(Y_{ij}-\hat m_{0,ij})^2}{\hat m_{0,ij}} \texttildelow \chi^2(\nu)$ for $\hat m_{0,11} = n\theta^2$, $\hat m_{0,12} = \hat m_{0,21} = n\hat\theta(1-\hat\theta)$, $\hat m_{0,22} = n(1-\hat\theta)^2$ and $\nu = dim(\Omega) - dim(\Omega_0) = 3-1 = 2$.
\\

d. $n = 251 + 34 + 48 + 5 = 338$

$\hat\theta = (2(251) + 34 + 48)/(2n) = 0.8639$

$S^2 = \frac{(251-338(0.8639)^2)^2}{338(0.8639)^2} + \frac{(34 - 338(0.8639)(1-0.8639))^2}{338(0.8639)(1-0.8639)} + \frac{(48 - 338(0.8639)(1-0.8639))^2}{338(0.8639)(1-0.8639)} + \frac{(5-338(1-0.8639)^2)^2}{338(1-0.8639)^2} = 2.806$

Using R, we find that $P(\chi^2(\nu = 2) > 2.806) = 0.2458583 > 0.05$, so we conclude that the free throws are plausibly independent and identically distributed.

\end{document}




